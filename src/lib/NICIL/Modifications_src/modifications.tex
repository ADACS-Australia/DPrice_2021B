% This is a sample LaTeX input file.
%
\documentclass{pasa}%
\usepackage{epstopdf}       % to convert eps images to pdf images
\usepackage{verbatim}       % to allow multiline comments
\usepackage{bm}		  % allow shorthand to bold text
\usepackage{aas_macros}  % to allow journal shortcuts in the bibliography

\newcommand{\nicil}{{\sc Nicil}}
\newcommand{\nicilf}{{\tt nicil.F90}}
\newcommand{\nicilsource}{{\tt nicil\_source.F90}}
\newcommand{\nicileta}{{\tt nicil\_ex\_eta.F90}}
\newcommand{\niciletasup}{{\tt nicil\_ex\_eta\_sup.F90}}
\newcommand{\nicilsph}{{\tt nicil\_ex\_sph.F90}}
\newcommand{\nicilsphsup}{{\tt nicil\_ex\_sph\_sup.F90}}

\title[NICIL]{NICIL: Important modifications}
\author[James Wurster]{James Wurster$^{1,2}$\thanks{j.wurster@exeter.ac.uk}\\
\affil{$^1$School of Physics, University of Exeter, Stocker Rd, Exeter EX4 4QL, UK}
\affil{$^2$Monash Centre for Astrophysics and School of Physics and Astronomy, Monash University, Vic 3800, Australia}}
%%
\jid{PASA}
\doi{10.1017/pas.\the\year.xxx}
\jyear{\the\year}


\usepackage[authoryear]{natbib}

\begin{document}%
%
\begin{abstract}
This document provides changes made to {\sc NICIL} since the publication \citet{Wurster2016}: \emph{NICIL: A stand alone library to self-consistently calculate non-ideal magnetohydrodynamic coefficients in molecular cloud cores}.
\end{abstract}
%
\begin{keywords}
methods: numerical -- (magnetohydrodynamics) MHD
\end{keywords}
\maketitle
%------------------------------------------------------------------------------------------------------------------------------------------------------------------------------------------------------------------------------------------------------------------------------------------------------------
\section{INTRODUCTION }
\label{sec:intro}
{\sc NICIL} is an ever-evolving code, thus, certain descriptions given in \citet{Wurster2016} will become outdated.  This document describes the important changes made since the publication of \citet{Wurster2016}; readers are encouraged to read \citet{Wurster2016} to obtain a full understanding of {\sc NICIL}, its usage, and installation.

The changes described here will be in chronological order.
%------------------------------------------------------------------------------------------------------------------------------------------------------------------------------------------------------------------------------------------------------------------------------------------------------------
\section{CHANGES WITHIN THE DESIGNATION OF \emph{VERSION 1.2.1}}
\subsection{Conservation of energy}
There is an error in (36) of \citet{Wurster2016}.  The equation should read
\begin{equation}
\left.\frac{\text{d} u}{\text{d} t}\right|_\text{OR} =  \frac{\eta_\text{OR}}{\rho}\bm{J}\cdot\bm{J} = \frac{\eta_\text{OR}}{\rho} J^2, \label{eq:u:ohm} \\
\end{equation}
That is, this equation should state $\left.\frac{\text{d} u}{\text{d} t}\right|_\text{OR} \geq 0$ as written here, not $\leq 0$ as in the paper.  Note that this error was in the paper only, and never existed within the code.

\subsection{Using Constant Coefficients}
This change was added to the repository on 8 November 2016.

There are currently two forms of constant coefficients for the non-ideal MHD terms.  They are 
\begin{subequations}
\label{eq:physical}
\begin{eqnarray}
\eta_\text{OR} &=&  \frac{m_\text{e}c^2}{4\pi e^2 n_\text{e,0}}, \\
\eta_\text{HE} &=&  s_\text{H} \frac{c}{4\pi e n_\text{e,0}}B, \\
\eta_\text{AD} &=&  \frac{1}{4\pi \gamma_\text{AD} \rho_\text{i,0}\left(\frac{\rho_\text{n}}{\rho_\text{n,0}}\right)^\alpha}v_\text{A}^2,
\end{eqnarray}
\end{subequations}
and
\begin{subequations}
\label{eq:semiconstant}
\begin{eqnarray}
\eta_\text{OR} &=&  C_\text{OR}, \\
\eta_\text{HE} &=&  C_\text{HE}B, \\
\eta_\text{AD} &=&  C_\text{AD}v_\text{A}^2 ,
\end{eqnarray}
\end{subequations}
The first option allows the user to set physical values to be used to calculation (e.g. as in \citet{PandeyWardle2008}), while the second option sets the entire prefix except for easily calculated values $B$ and $v_\text{A}$ (e.g. as in Appendix C.1.2 of \citet{WPB2016}).  We now add a third option:
\begin{subequations}
\label{eq:constant}
\begin{eqnarray}
\eta_\text{OR} &=&  C_\text{OR}, \\
\eta_\text{HE} &=&  C_\text{HE}, \\
\eta_\text{AD} &=&  C_\text{AD},
\end{eqnarray}
\end{subequations}
where the values are constant for all time.

In terms of the {\sc NICIL} code, the logical {\tt eta\_const\_calc} has been removed, and replaced with the integer {\tt eta\_const\_type}, which can currently take three values:
\begin{itemize}
\item  {\tt icnstphys}$(\equiv 1)$, which uses equation \eqref{eq:physical}
\item  {\tt cnstsemi}$(\equiv 2)$, which uses equation \eqref{eq:semiconstant}
\item  {\tt icnst}        $(\equiv 3)$, which uses equation \eqref{eq:constant}
\end{itemize}
As with {\tt eta\_const\_calc}, {\tt eta\_const\_type} is not used if {\tt eta\_const=.False.}

\subsection{Modification when calculating grain charges using \texttt{nicil\_ionR\_get\_n\_via\_Zave}}
In the subroutine \texttt{nicil\_ionR\_get\_n\_via\_Zave} (Appendix D of \citealp{Wurster2016}), the grain number densities are calculated via
\begin{eqnarray}
n_\text{g}(Z=-1,a_j) &=& \frac{k_\text{eg}^+n_\text{e}(1+\bar{Z})n_\text{g}}{ \sum_s k_{s\text{g}}^-n_s +  k_\text{eg}^-n_\text{e} + 2k_\text{eg}^0n_\text{e}} \label{eq:ngneg}\\
n_\text{g}(Z=+1,a_j) &=& \frac{(1-\bar{Z})n_\text{g}\sum_s k_{s\text{g}}^0 n_s }{ \sum_s \left[k_{s\text{g}}^+ +2k_{s\text{g}}^0\right]n_s + k_\text{eg}^+ n_\text{e} } \label{eq:ngpos}
\end{eqnarray}
where $k^+\equiv k(Z=1,a_j)$, $k^-\equiv k(Z=-1,a_j)$ and $k^0\equiv k(Z=0,a_j)$.  However, given the assumptions made in this section, these results do not guarantee charge neutrality.  Thus, we recalculate \eqref{eq:ngpos} as an average of it and its value using \eqref{eq:ngneg} when assuming charge neutrality:
\begin{eqnarray}
n_\text{g}(Z=1,a_j) &=&  \frac{1}{2}\frac{(1-\bar{Z})n_\text{g}\sum_s k_{s\text{g}}^0 n_s }{ \sum_s \left[k_{s\text{g}}^+ +2k_{s\text{g}}^0\right]n_s + k_\text{eg}^+ n_\text{e} }  \notag \\
&+& \frac{1}{2}\left(\frac{k_\text{eg}^+n_\text{e}(1+\bar{Z})n_\text{g}}{ \sum_s k_{s\text{g}}^-n_s +  k_\text{eg}^-n_\text{e} + 2k_\text{eg}^0n_\text{e}}\right. \notag \\
&+& \left.n_\text{e} -\sum_s n_s  \right). \label{eq:ngposnew}
\end{eqnarray}
Using the revised value of $n_\text{g}(Z=1,a_j)$ from \eqref{eq:ngposnew}, we obtain
\begin{eqnarray}
n_\text{g}(Z=-1,a_j) &=& \sum_s n_s  + n_\text{g}(Z=1,a_j) - n_\text{e}.
\end{eqnarray}


This change was added to the repository on 6 January 2017.

%------------------------------------------------------------------------------------------------------------------------------------------------------------------------------------------------------------------------------------------------------------------------------------------------------------
\bibliographystyle{apj}
\bibliography{Wbib_nicil.bib}


\end{document}
